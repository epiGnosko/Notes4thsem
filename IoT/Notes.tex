\documentclass[11pt,letterpaper]{article}

\usepackage{hyperref}
\usepackage{graphicx}
\usepackage{fancybox}
\usepackage[utf8]{inputenc}
\usepackage{epsfig,graphicx}
\usepackage{multicol,pst-plot}
\usepackage{pstricks}
\usepackage{amsmath}
\usepackage{amsfonts}
\usepackage{amssymb}
\usepackage{eucal}
\usepackage{upgreek}
\usepackage[left=2cm,right=2cm,top=2cm,bottom=1cm]{geometry}
\usepackage{tcolorbox}
\usepackage{import}
\pagestyle{empty}
\DeclareMathOperator{\tr}{Tr}
\renewcommand{\sp}[1]{$${\begin{split}#1\end{split}}$$}

\usepackage{lipsum}
\usepackage{mdframed}
\usepackage{listings}
\usepackage{color}

% Margins
% \topmargin=-0.45in
\evensidemargin=0in
\oddsidemargin=0in
\textwidth=6.5in
\textheight=9.0in
\headsep=0.25in

 % The problem environment introduced.                                     
\newenvironment{problem}[2][Problem]                                  
        {\begin{tcolorbox}[colback=white,colframe=gray!50,title=#1 #2]}
        {\end{tcolorbox}}
        % {\begin{mdframed}[backgroundcolor=gray!20] \textbf{#1 #2} \\}
        % {\end{mdframed}}
% Define solution environment
\newenvironment{solution}                      
        {\begin{mdframed}\textit{Solution:} \\}
        {\end{mdframed}}
% Define an environments for proofs
\newenvironment{myproof} 
        {\textit{Proof:}}                                   
        {\begin{flushright} Q.E.D. \end{flushright}}
% Define a theorem environment and a notation one too
\newenvironment{mytheorem}                    
        {\begin{mdframed}\textbf{Theorem:} \\}
        {\end{mdframed}}
\newenvironment{notation}                      
        {\begin{mdframed}\textit{Notation:} \\}
        {\end{mdframed}}
% A new example wouldnt so any harm either...  
\newenvironment{example}                             
        {\textit{Example:}\\}
	{}
%I should be ashamed to forget the definition environment
\newenvironment{definition}
	{\begin{mdframed}$\underline{\textit{Def}^\textit{n}:} $\\}
	{\end{mdframed}}
%Corollary envvvvvvvvv
\newenvironment{corollary}
	{\textbf{Corrolary:}\\}

\pagestyle{empty}

\begin{document}

\begin{center}
  \Huge{JAVA Notes}\\
  \vspace{0.25cm}
  \small{Gurmukh Singh}
\end{center}

\vspace{-1.75cm}

\begin{flushright}
  Instructor: \\ Dr. Silky
\end{flushright}

\vspace{-1.3cm}

\begin{flushleft}
  B.Tech. CSE
\end{flushleft}

\rule{15.5cm}{0.1mm}%{\linewidth}{0.1mm}

% Optional TOC
\tableofcontents
\pagebreak

%--Paper--

\section{Introduction}
\begin{itemize}
  \item IOT refers nto a system of iterrelated, internet connected objects that are able to collect and transfer data over a wireless channel without humna intervention. 
  \item IoT describes the network of physical objects - "things" - that are embedded with sensors, software, and other techonlogies. 
  \item Insights aid in transforming business and lower costs with aid if reliable real time data through, 
    \begin{itemize}
      \item Reduction of wasted materials. 
      \item Streamlined Oprational and mechanical processes. 
      \item Expansion.
    \end{itemize}
\end{itemize}

\subsection{How it works}
\begin{enumerate}
  \item Sensors/DEvices collect data
  \item Data Processing makes the data useful. 
  \item Connectivity sends the data to cloud. 
  \item UI delivers the intel to the user. 
\end{enumerate}

\subsection{IOT Architecture}
\begin{enumerate}
  \item Things, sensors, controllers. 
  \item Gateways and data acquisition 
  \item Edge analytics. 
  \item Data centre/ cloud platform
\end{enumerate}

\subsection{Layers in IoT Architecture}
\begin{itemize}
  \item Data management layer
  \item Application layer
  \item Network layer
  \item Perception layer
\end{itemize}

\subsection{Advantages of IoT}
\begin{itemize}
  \item Efficient resource utilization.
\end{itemize}
\end{document}
